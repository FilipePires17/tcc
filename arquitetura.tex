\section{Requisitos e Arquitetura}

\subsection{Análise de contexto}

\subsubsection{Visão geral}

    O sistema proposto é um aplicativo para dispositivos móveis que visa facilitar a navegação pelo campus da UFMS, em Campo Grande.  A principal responsabilidade do aplicativo é fornecer informações sobre a localização de prédios, serviços e pontos de interesse no campus, bem como permitir que usuários sugiram adições de novos pontos de interesse e atualizações de informações.

\subsubsection{Condições Restritivas}

    O aplicativo é projetado exclusivamente para dispositivos móveis com sistema operacional Android, tendo o Android 6.0 (API 23) como versão mínima suportada. O aplicativo também exige conexão com a internet para funcionar corretamente.

\subsubsection{Benefícios}

    O aplicativo auxiliará estudantes, professores, visitantes e demais usuários do campus a se localizarem e a encontrarem informações sobre os prédios e serviços disponíveis. Além disso, o aplicativo permitirá que usuários sugiram adições de novos pontos de interesse e atualizações de informações, contribuindo para a melhoria contínua do aplicativo.

\subsection{Requisitos}