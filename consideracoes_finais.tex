\section{Considerações Finais}
\label{sec:consideracoes_finais}

    O objetivo do aplicativo Encontre na UFMS foi desde o ínicio ser uma ferramenta simples e acessível para ajudar os frequentadores do campus a se localizarem no dia a dia. Conseguimos implementar isso graças a integração com a API do Google Maps em conjunto com as funcionalidades de colaboração como a criação de novos locais e edição de informações que possam estar defasadas. Acreditamos que a aplicação pode ser muito útil para os estudantes, professores e visitantes da UFMS, facilitando a navegação pelo campus e tornando a experiência mais agradável.

    A partir do uso do Flutter para desenvolvimento do aplicativo, foi possível desenvolver um aplicativo com uma interface amigável e intuitiva e que no futuro pode ser expandida para mais plataformas além do Android. A integração com o backend foi simples e eficiente graças a adoção do Fastify com Node.js como servidor. No banco de dados o MySQL foi uma escolha acertada tendo em vista a natureza relacional dos dados presentes no aplicativo, além de ser uma ferramenta amplamente utilizada e com muita documentação disponível.

    A motivação para o desenvolvimento do aplicativo foram as muitas vezes em que um visitante ou calouro veio nos perguntar onde ficava um determinado prédio ou sala e na maior parte das vezes não sabíamos responder e não sabíamos onde encontrar aquela informação. Com o Encontre na UFMS esperamos que essas situações sejam menos frequentes e que a comunidade acadêmica possa se beneficiar da ferramenta. No começo a quantidade de locais cadastrados não será grande, mas após um tempo e com o apoio da comunidade acadêmica do campus acreditamos que o aplicativo pode se tornar uma ferramenta muito útil para todos.

    As aplicações já mencionadas, Google Maps e Localização UFMS, são ótimas ferramentas mas que não abrangem todas as necessidades da comunidade acadêmica da UFMS. O Google Maps é uma ferramenta muito ampla e que não é focada em um único local, é difícil encontrar diversas localidades importantes nela. Já o Localização UFMS possui quase todas as localidades da UFMS, mas não possui uma interface amigável e intuitiva como o Google Maps e não possui funcionalidades de traçar rotas e navegação, e por não ser colaborativo pode ter informações desatualizadas ou faltantes.

    O aplicativo Encontre na UFMS é uma ferramenta que tenta unir o melhor dos dois mundos, com uma interface amigável e intuitiva como o Google Maps e com a colaboratividade e foco na UFMS do Localização UFMS. Acreditamos que o aplicativo pode ser muito útil para a comunidade acadêmica da UFMS e que com o tempo e o apoio da comunidade ele pode trazer uma qualidade de vida maior aos transeuntes do campus.