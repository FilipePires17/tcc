\section{Introdução}
O campus de Campo Grande da UFMS conta com uma vasta área que é ocupada por diversas faculdades, institutos e blocos em geral. Este fato dificulta que visitantes, alunos e até mesmo docentes conheçam ou encontrem locais presentes na UFMS. Dessa forma, foi criado o Encontre na UFMS, um aplicativo que visa centralizar locais e pontos de interesse no campus da UFMS de Campo Grande.

O principal objetivo do aplicativo é permitir buscar e localizar pontos de interesse dentro do campus, de modo que o usuário consiga informações de como chegar ao local, além de outras informações que possam ser pertinentes para conhecimento geral e do indivíduo. O aplicativo tem o intuito de ser cooperativo para a comunidade acadêmica, de forma que todos possam contribuir para atualização constante do mesmo. Além de permitir ao usuário facilidade para acesso e também utilização.

Este artigo está estruturado da seguinte maneira: Seção 2: Trabalhos Relacionados - Nesta seção, abordamos aplicativos existentes que têm objetivos semelhantes ao do aplicativo Encontre na UFMS e que podem fornecer insights úteis para seu desenvolvimento e implementação. Dois aplicativos relevantes nesse contexto são Google Maps[x] da Google LLC e Localização UFMS[x] da UFMS. Seção 3: Referencial Teórico - Nesta seção, apresentaremos uma visão geral dos principais conceitos e tecnologias utilizadas no desenvolvimento do aplicativo Encontre na UFMS, abrangendo desde o ambiente de backend até a interface do usuário no frontend.