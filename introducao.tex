\section{Introdução}
O campus de Campo Grande da UFMS possui uma vasta área ocupada por diversas faculdades, institutos e blocos em geral. Esse fato dificulta que visitantes, alunos e até mesmo docentes conheçam ou encontrem locais presentes na UFMS. Dessa forma, foi criado o Encontre na UFMS, um aplicativo que visa centralizar locais e pontos de interesse no campus da UFMS de Campo Grande.

O principal objetivo do aplicativo é permitir que os usuários busquem e localizem pontos de interesse dentro do campus, de modo que obtenham informações sobre como chegar ao local, além de outras informações que possam ser pertinentes para conhecimento geral e individual. O aplicativo tem o intuito de ser colaborativo para a comunidade acadêmica, possibilitando que todos contribuam para sua constante atualização, além de proporcionar facilidade de acesso e utilização aos usuários.

Este artigo está estruturado da seguinte maneira: Seção 2: Nesta seção foi abordado os trabalhos relacionados. Seção 3: Nesta seção foi abordado o referencial teórico para o desenvolvimento do aplicativo. Seção 4: Nesta seção foi abordado a implementação. Seção 5: Considerações finais. Seção 6: Nesta seção foi abordado os trabalhos futuros. Referências: Nesta seção foram listadas as referências utilizadas no desenvolvimento deste trabalho.