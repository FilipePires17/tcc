\section{Introdução}
O campus da UFMS de Campo Grande possui uma vasta área ocupada por diversas faculdades, institutos e blocos em geral. Esse fato dificulta que visitantes, alunos e até mesmo docentes conheçam ou encontrem locais presentes na UFMS. Dessa forma, foi criado o Encontre na UFMS, um aplicativo que visa centralizar locais e pontos de interesse no campus da UFMS de Campo Grande.

O principal objetivo do aplicativo é permitir que os usuários busquem e localizem pontos de interesse dentro do campus, de modo que obtenham informações sobre como chegar ao local, além de outras informações que possam ser pertinentes para conhecimento geral e individual. O aplicativo tem o intuito de ser colaborativo para a comunidade acadêmica, possibilitando que todos contribuam para sua constante atualização, além de proporcionar facilidade de acesso e utilização aos usuários.

Este artigo está estruturado da seguinte maneira: \hyperref[sec:trabalhos_relacionados]{Seção 2}: Nesta seção são abordados os trabalhos relacionados. \hyperref[sec:referencial_teorico]{Seção 3}: Nesta seção é abordado o referencial teórico para o desenvolvimento do aplicativo. \hyperref[sec:arquitetura]{Seção 4}: Nesta seção são abordados os requisitos e a arquitetura. \hyperref[sec:implementacao]{Seção 5}: Nesta seção é abordada a implementação. \hyperref[sec:consideracoes_finais]{Seção 6}: Esta seção apresenta as considerações finais. \hyperref[sec:trabalhos_futuros]{Seção 7}: Nesta seção são abordados os trabalhos futuros. \hyperref[sec:referencias]{Referências}: Nesta seção são listadas as referências utilizadas no desenvolvimento deste trabalho.