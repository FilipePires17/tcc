\section{Referencial Teórico}

Nesta seção, apresentaremos uma visão geral dos principais conceitos e tecnologias utilizadas no desenvolvimento do aplicativo Encontre na UFMS, abrangendo desde os requisitos do sistema até o ambiente do backend e a interface do usuário no frontend.

\subsection{Requisitos do Sistema}

O aplicativo Encontre na UFMS foi planejado com base em um conjunto de requisitos, focando na experiência do usuário e considerando aspectos como facilidade de uso, acessibilidade e usabilidade. Abaixo, estão listados alguns dos requisitos funcionais descritos no Documento de Requisitos Aplicativo Encontre na UFMS [3]:

\begin{itemize}
  \item \textbf{Listagem de pontos importantes}: O sistema deve listar pontos importantes do campus de Campo Grande da UFMS. Exemplo: Blocos Acadêmicos, Bancos, Pontos Turísticos, Restaurantes, etc.
  \item \textbf{Listagem de informações detalhadas}: O sistema deve fornecer informações detalhadas a respeito de um determinado local, tais como: Nome, endereço, telefone, horário de funcionamento, fotos, entre outras informações.
  \item \textbf{Redirecionamento para o Google Maps}:  O sistema deve oferecer um link que redirecione o usuário para o aplicativo ou página na internet do Google Maps com o endereço do local inserido.
  \item \textbf{Busca de pontos específicos}: O sistema deve permitir que o usuário busque um ponto específico utilizando o nome do local e/ou filtros de busca.
  \item \textbf{Login}: O sistema deve permitir que o usuário faça login, além de possibilitar o cadastro e a recuperação de conta.
  \item \textbf{Favoritar}:  O sistema deve permitir que o usuário favorite os locais de interesse.
  \item \textbf{Avaliar}:  O sistema deve permitir que o usuário avalie os locais, utilizando uma escala de 1 a 5 estrelas.
  \item \textbf{Criar e alterar locais}: O sistema deve permitir que usuários façam sugestões para inclusão ou alteração de locais.
\end{itemize}

O sistema visa oferecer ao usuário uma experiência de busca e localização eficiente, com uma interface amigável e intuitiva que permite navegar e encontrar informações úteis sobre cada local. Além disso, possui funcionalidades como redirecionamento para o Google Maps, a fim de traçar rotas para que o usuário consiga chegar ao local. Também inclui funcionalidades de cadastro, login e recuperação de conta, permitindo ao usuário usufruir de opções como avaliação de locais, favoritá-los e sugerir novos locais ou editar locais existentes no aplicativo.


\subsection{Backend: Fastify, Drizzle e MySQL}
O backend foi desenvolvido utilizando o framework Fastify com Node.js, proporcionando alto desempenho e grande velocidade nas requisições em comparação a outros frameworks, além da facilidade de uso. Foi utilizado TypeScript para aumentar a segurança contra erros de tipagem durante o processo de desenvolvimento.

Para o armazenamento de dados no backend, foi utilizado o MySQL, um sistema de gerenciamento de banco de dados relacional open-source que oferece confiabilidade e desempenho. A conexão entre o Node.js e o MySQL foi feita utilizando o Drizzle, um mapeador de objetos (ORM) que foca na velocidade e eficiência na entrega de dados entre as duas ferramentas.

\subsection{Login}
Os usuários tem a possibilidade de se cadastrarem no aplicativo, sendo o cadastro um requisito não obrigatório. Para usuários logados no sistema, são utilizados tokens JWT (JSON Web Token) para autenticação, garantindo acesso seguro e eficaz ao sistema.

Para o envio de e-mails de recuperação de senha, foi utilizado o MailJet, uma ferramenta que fornece acesso a APIs para envio de e-mails e mensagens de texto (SMS).

\subsection{Frontend: Flutter}
O frontend foi desenvolvido utilizando a ferramenta Flutter, um framework para a linguagem Dart e que pertence ao Google. Tanto o Dart quanto o Flutter foram criados principalmente para possibilitar o desenvolvimento de aplicativos móveis. Por se tratar de uma ferramenta relativamente nova, possui apenas 7 anos desde o seu lançamento, ela se aproveita de vários padrões já bem estabelecidos em outras linguagens e frameworks para criar um ambiente moderno e de fácil adequação para os desenvolvedores.

Uma das grandes preocupações ao se desenvolver um aplicativo móvel é a gerência de possíveis estados enquanto o usuário navega e interage com este aplicativo, como fazer telas conversarem entre si e como manter a experiência o mais contínua e suave possível. Para isso é muito importante separar a apresentação dos dados para o usuário da lógica de negócios que faz todo o tratamento desses dados. A fim de exercer esse papel foi utilizdo o padrão BLoC (Business Logic Component), com a biblioteca flutter\_bloc, que permite acessar todo o estado da aplicação a qualquer momento e também dispara eventos quando certos estados mudam para que a apresentação dos dados esteja sempre atualizada.

Para armazenar localmente o token do usuário logado foi utilizado o Hive, um banco de dados não relacional que tem como principais características a leveza e a rapidez. O mini mapa que é utilizado para mostrar a localização e para cadastrar um novo local facilitando o processo para o usuário foi implementado utilizando a biblioteca google\_maps\_flutter desenvolvido pela própria equipe do Flutter. 

\subsection{Ambiente de desenvolvimento: Visual Studio Code}
O ambiente de desenvolvimento escolhido foi o Visual Studio Code, uma das IDEs mas utilizadas no mundo graças ao grande número de ferramentas que podem ser integradas nela para aumentar as funcionalidades de desenvolvimento, teste e depuração.

Foi utilizado também o Android SDK (Software Development Kit), que é um conjunto de ferramentas que permite o desenvolvimento de aplicativos para a plataforma Android. Ele inclui um depurador, bibliotecas, um emulador de dispositivo baseado em QEMU, documentação, amostras de código e tutoriais.

\subsection{Bibiliotecas e Dependências}
Diversas bibliotecas e dependências foram utilizadas no desenvolvimento do aplicativo Encontre na UFMS, dentre elas destacam-se:

\begin{itemize}
  \item \textbf{Fastify}: Framework para desenvolvimento de aplicações web com Node.js.
  \item \textbf{TypeScript}: Linguagem de programação que extende JavaScript, adicionando tipagem à linguagem.
  \item \textbf{tsup}: Empacotador de arquivos TypeScript e JavaScript, utilizado para otimizar e gerar builds.
  \item \textbf{MySQL}: MySQL é um sistema de gerenciamento de banco de dados relacional de código aberto baseado em SQL.
  \item \textbf{Drizzle}: Mapeador de objetos para conexão entre Node.js e MySQL.
  \item \textbf{MailJet}: Ferramenta para envio de emails e sms.
  \item \textbf{Flutter}: Framework para desenvolvimento de aplicativos móveis.
  \item \textbf{Dart}: Linguagem de programação utilizada no Flutter.
  \item \textbf{hive}: Banco de dados não relacional utilizado para armazenamento local.
  \item \textbf{google\_maps\_flutter}: Biblioteca para implementação de mapas no Flutter.
  \item \textbf{flutter\_bloc}: Biblioteca para gerenciamento de estados no Flutter.
  \item \textbf{dio}: Biblioteca para requisições HTTP no Flutter.
\end{itemize}

Todas essas bibliotecas e dependências, bem como outras que não foram aqui citadas mas encontram-se especificadas nos projetos do frontend e do backend, foram fundamentais para o desenvolvimento do aplicativo Encontre na UFMS, permitindo a implementação de funcionalidades essenciais e a integração entre o backend e o frontend.