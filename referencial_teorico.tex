\section{Referencial Teórico}

Nesta seção, apresentaremos uma visão geral dos principais conceitos e tecnologias utilizadas no desenvolvimento do aplicativo Encontre na UFMS, abrangendo desde o backend até a interface do usuário no frontend.

\subsection{Backend: Fastify, Drizzle e MySQL}
O backend foi desenvolvido utilizando o framework \textit{Fastify} \cite{fastify} com \textit{Node.js} \cite{nodejs}, proporcionando alto desempenho e grande velocidade nas requisições em comparação a outros frameworks, além da facilidade de uso. Foi utilizado \textit{TypeScript} \cite{typescript} para aumentar a segurança contra erros de tipagem durante o processo de desenvolvimento.

Para o armazenamento de dados no backend, foi utilizado o \textit{MySQL} \cite{mysql}, um sistema de gerenciamento de banco de dados relacional open-source que oferece confiabilidade e desempenho. A conexão entre o Node.js e o MySQL foi feita utilizando o \textit{Drizzle} \cite{drizzle}, um mapeador de objetos (ORM) que foca na velocidade e eficiência na entrega de dados entre as duas ferramentas.

\subsection{Login}
Os usuários tem a possibilidade de se cadastrarem no aplicativo, sendo o cadastro um requisito não obrigatório. Para usuários logados no sistema, são utilizados tokens \textit{JWT (JSON Web Token)} \cite{jwt} para autenticação, garantindo acesso seguro e eficaz ao sistema.

Para o envio de e-mails de recuperação de senha, foi utilizado o \textit{MailJet} \cite{mailjet}, uma ferramenta que fornece acesso a APIs para envio de e-mails e mensagens de texto (SMS).

\subsection{Frontend: Flutter}
O frontend foi desenvolvido utilizando a ferramenta \textit{Flutter} \cite{flutter}, um framework para a linguagem \textit{Dart} \cite{dart} e que pertence ao Google. Tanto o Dart quanto o Flutter foram criados principalmente para possibilitar o desenvolvimento de aplicativos móveis. Por se tratar de uma ferramenta relativamente nova, possui apenas 7 anos desde o seu lançamento, ela se aproveita de vários padrões já bem estabelecidos em outras linguagens e frameworks para criar um ambiente moderno e de fácil adequação para os desenvolvedores.

Uma das grandes preocupações ao se desenvolver um aplicativo móvel é a gerência de possíveis estados enquanto o usuário navega e interage com este aplicativo, como fazer telas conversarem entre si e como manter a experiência o mais contínua e suave possível. Para isso é muito importante separar a apresentação dos dados para o usuário da lógica de negócios que faz todo o tratamento desses dados. A fim de exercer esse papel foi utilizdo o padrão BLoC (Business Logic Component), com a biblioteca \textit{flutter\_bloc} \cite{flutterbloc}, que permite acessar todo o estado da aplicação a qualquer momento e também dispara eventos quando certos estados mudam para que a apresentação dos dados esteja sempre atualizada.

Para armazenar localmente o token do usuário logado foi utilizado o \textit{Hive} \cite{hive}, um banco de dados não relacional que tem como principais características a leveza e a rapidez. O mini mapa que é utilizado para mostrar a localização e para cadastrar um novo local facilitando o processo para o usuário foi implementado utilizando a biblioteca \textit{google\_maps\_flutter} \cite{googlemapsflutter} desenvolvido pela própria equipe do Flutter. 

\subsection{Ambiente de desenvolvimento: Visual Studio Code}
O ambiente de desenvolvimento escolhido foi o \textit{Visual Studio Code} \cite{visualstudiocode}, uma das IDEs mas utilizadas no mundo graças ao grande número de ferramentas que podem ser integradas nela para aumentar as funcionalidades de desenvolvimento, teste e depuração.

Foi utilizado também o \textit{Android SDK (Software Development Kit)} \cite{androidsdk}, que é um conjunto de ferramentas que permite o desenvolvimento de aplicativos para a plataforma Android. Ele inclui um depurador, bibliotecas, um emulador de dispositivo baseado em \textit{QEMU} \cite{QEMU}, documentação, amostras de código e tutoriais.

\subsection{Bibiliotecas e Dependências}
Diversas bibliotecas e dependências foram utilizadas no desenvolvimento do aplicativo Encontre na UFMS, dentre elas destacam-se:

\begin{itemize}
  \item \textbf{Fastify}: Framework para desenvolvimento de aplicações web com Node.js \cite{fastify}.
  \item \textbf{TypeScript}: Linguagem de programação que extende JavaScript, adicionando tipagem à linguagem \cite{typescript}.
  \item \textbf{tsup}: Empacotador de arquivos TypeScript e JavaScript, utilizado para otimizar e gerar builds \cite{tsup}.
  \item \textbf{MySQL}: MySQL é um sistema de gerenciamento de banco de dados relacional de código aberto baseado em SQL \cite{mysql}.
  \item \textbf{Drizzle}: Mapeador de objetos para conexão entre Node.js e MySQL \cite{drizzle}.
  \item \textbf{MailJet}: Ferramenta para envio de emails e sms \cite{mailjet}.
  \item \textbf{Flutter}: Framework para desenvolvimento de aplicativos móveis \cite{flutter}.
  \item \textbf{Dart}: Linguagem de programação utilizada no Flutter \cite{dart}.
  \item \textbf{hive}: Banco de dados não relacional utilizado para armazenamento local \cite{hive}.
  \item \textbf{google\_maps\_flutter}: Biblioteca para implementação de mapas no Flutter \cite{googlemapsflutter}.
  \item \textbf{flutter\_bloc}: Biblioteca para gerenciamento de estados no Flutter \cite{flutterbloc}.
  \item \textbf{dio}: Biblioteca para requisições HTTP no Flutter \cite{dio}.
\end{itemize}

Todas essas bibliotecas e dependências, bem como outras que foram especificadas nos projetos do frontend e do backend, foram fundamentais para o desenvolvimento do aplicativo Encontre na UFMS, permitindo a implementação de funcionalidades essenciais e a integração entre o backend e o frontend.