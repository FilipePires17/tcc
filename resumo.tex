  \begin{resumo} 
    Este artigo descreve o desenvolvimento do aplicativo móvel ``Encontre na UFMS'', que surgiu com o intuito de suprir a falta de acesso às informações dos locais presentes no campus da UFMS de Campo Grande, para alunos, visitantes e docentes. O principal objetivo do aplicativo Encontre na UFMS é oferecer uma ferramenta simples e acessível para ajudar os seus usuários a encontrarem pontos de interesse, além de possibilitar a cooperatividade da comunidade de usuários que podem sugerir a criação ou alteração de locais presentes no aplicativo, possibilitando que o aplicativo mantenha-se sempre atualizado. O aplicativo foi desenvolvido utilizando: o framework Flutter no frontend, o framework Fastify no backend e o banco de dados MySQL para armazenamento de dados gerais do aplicativo. Este artigo detalha a estrutura e o funcionamento do aplicativo.
  \end{resumo}
  
  \begin{abstract}
    This article describes the development of the mobile application ``Encontre na UFMS'', which emerged with the aim of filling the lack of access to information about the locations on the UFMS campus in Campo Grande, for students, visitors and teachers. The main objective of the Encontre na UFMS application is to offer a simple and accessible tool to help its users find points of interest, as well as to enable the cooperativity of the user community that can suggest the creation or alteration of locations present in the application, enabling the application to always stay up to date. The application was developed using: the Flutter framework on the frontend, the Fastify framework on the backend and the MySQL database for storing general application data. This article details the structure and operation of the application.
  \end{abstract}

  \newpage