  \begin{resumo} 
    Este artigo descreve o desenvolvimento do aplicativo móvel ``Encontre na UFMS'', que surgiu devido a falta de acessibilidade de informações dos locais presentes no campus de Campo Grande da UFMS, para alunos, visitantes e docentes que fosse concentrados em uma aplicação. O principal objetivo do aplicativo Encontre na UFMS é oferecer uma ferramenta simples e acessível para ajudar os seus usuários a encontrarem pontos de interesse, além de possibilitar a cooperatividade da comunidade de usuários que podem sugerir a criação ou alteração de locais presentes no aplicativo, possibilitando que o aplicativo mantenha-se sempre atualizado. O aplicativo foi desenvolvido utilizando: o framework Flutter no frontend, o framework Fastify no backend e o banco de dados MySQL para armazenamento de dados gerais do aplicativo. Este artigo detalha a estrutura e o funcionamento do aplicativo.
  \end{resumo}
  
  \begin{abstract}
    This article describes the development of the mobile application ``Encontre na UFMS'', which was created due to the lack of accessible information about locations on the Campo Grande campus of UFMS, aimed at students, visitors, and faculty, all concentrated in a single application. The main goal of the Encontre na UFMS app is to offer a simple and accessible tool to help users find points of interest, while also enabling community cooperation, where users can suggest the creation or modification of locations within the app, keeping it up to date. The application was developed using the Flutter framework for the frontend, the Fastify framework for the backend, and MySQL for storing general application data. This article details the app's structure and functionality.
  \end{abstract}

  \newpage