\section{Solução Implementada}
\label{sec:implementacao}

    O aplicativo Encontre na UFMS foi desenvolvido com o intuito de facilitar a navegação pelo campus da UFMS em Campo Grande, permitindo que usuários encontrem e localizem pontos de interesse dentro do campus, estes sendo cadastrados por outros usuários, possibilitando que locais que não estão presentes em outros aplicativos de localização sejam documentados e compartilhados com a comunidade acadêmica. Os usuários também terão poder de avaliar os locais cadastrados, podendo assim, ajudar outros usuários a escolherem o melhor local para suas necessidades.

\subsection{Implementação}

    O desenvolvimento do aplicativo Encontre na UFMS foi dividido em duas partes principais: o backend e o frontend. Trataremos a seguir de cada uma dessas partes, detalhando as tecnologias utilizadas e as funcionalidades implementadas.

\subsubsection{Frontend}

    O frontend foi desenvolvido utilizando a ferramenta \textit{Flutter} \cite{flutter}, um framework para a linguagem \textit{Dart} \cite{dart} e que pertence ao Google. Tanto o Dart quanto o Flutter foram criados principalmente para possibilitar o desenvolvimento de aplicativos móveis. Por se tratar de uma ferramenta relativamente nova, possui apenas 7 anos desde o seu lançamento, ela se aproveita de vários padrões já bem estabelecidos em outras linguagens e frameworks para criar um ambiente moderno e de fácil adequação para os desenvolvedores.

    A arquitetura do aplicativo foi dividida em três camadas principais: a camada de apresentação, a camada de domínio e a camada de dados. A camada de apresentação é responsável por exibir os dados para o usuário e capturar as interações do usuário com o sistema. A camada de domínio contém as regras de negócio do sistema, enquanto a camada de dados é responsável por acessar os dados do sistema, seja de um banco de dados local ou de uma API remota. Essa divisão de camadas é ilustrada na Figura 5.

    \begin{figure}[h]
        \centering
        \includegraphics[width=0.5\textwidth]{imagens/cleanarch.png}
        \caption{\scriptsize Exemplificação da arquitetura em camadas}
        \footnotesize  \centering{\textbf{Fonte: Semih Altin\cite{cleanarchitecture}}}
        \label{fig:clean-architecture}
    \end{figure}

    \FloatBarrier

    Embora o Flutter seja capaz de gerar aplicativos nativos para Android e iOS, o aplicativo Encontre na UFMS foi desenvolvido apenas para Android, visto que a Apple, dona do sistema operacional iOS, restringe o desenvolvimento para seu sistema caso o desenvolvedor não possua um dispositivo da marca. Mas, em teoria, o aplicativo poderia ser facilmente adaptado para iOS, bastando apenas compilar o código fonte em um ambiente de desenvolvimento da Apple. O código-fonte do frontend está disponível no repositório em: \textit{Frontend Encontre Na Ufms} \cite{frontend}.

    Uma das grandes preocupações ao se desenvolver um aplicativo móvel é a gerência de possíveis estados enquanto o usuário navega e interage com este aplicativo, como fazer telas conversarem entre si e como manter a experiência o mais contínua e suave possível. Para isso é muito importante separar a apresentação dos dados para o usuário da lógica de negócios que faz todo o tratamento desses dados. A fim de exercer esse papel foi utilizdo o padrão BLoC (Business Logic Component), com a biblioteca \textit{flutter\_bloc} \cite{flutterbloc}, que permite acessar todo o estado da aplicação a qualquer momento e também dispara eventos quando certos estados mudam para que a apresentação dos dados esteja sempre atualizada.

    Para armazenar localmente o token do usuário logado foi utilizado o \textit{Hive} \cite{hive}, um banco de dados não relacional que tem como principais características a leveza e a rapidez. O mini mapa que é utilizado para mostrar a localização e para cadastrar um novo local facilitando o processo para o usuário foi implementado utilizando a biblioteca \textit{google\_maps\_flutter} \cite{googlemapsflutter} desenvolvido pela própria equipe do Flutter. 

\subsubsection{Backend}

    O backend foi desenvolvido utilizando o framework Fastify, a fim de garantir excelente desempenho mesmo com um grande volume de requisições, permitindo escalabilidade sem perder a segurança. Foi programado utilizando o Node.js, uma ferramenta que permite a execução de código JavaScript em um ambiente fora do navegador, com excelente desempenho e confiabilidade.
    % prefiro o segundo texto
    O backend foi desenvolvido utilizando o framework \textit{Fastify} \cite{fastify} com \textit{Node.js} \cite{nodejs}, proporcionando alto desempenho e grande velocidade nas requisições em comparação a outros frameworks, além da facilidade de uso. Foi utilizado \textit{TypeScript} \cite{typescript} para aumentar a segurança contra erros de tipagem durante o processo de desenvolvimento.
    
    Durante o desenvolvimento, foi utilizado a linguagem de programação Typescript para tipagem de variáveis, classes e objetos, evitando eventuais erros que poderiam ser causados por incompatibilidade de tipos. Para a execução do código utilizando o Node.js, foi utilizado uma ferramenta do TypeScript: "tsc" ou TypeScript Compiler, para transpilação do código-fonte em JavaScript durante o desenvolvimento, enquanto que para ser utilizado em produção, foi utilizado o tsup, uma ferramenta de empacotamento e transpilação de código TypeScript para JavaScript, rápida e eficiente.

    O código foi desenvolvido utilizando o padrão de software \textit{MVC ou Model-View-Controller} \cite{mvc}, sendo em português: Modelo-Visão-Controlador. As três camadas se entrelaçam entre si, onde o Controlador recebe as requisições do usuário e as envia para o Modelo, que manipula os dados e faz a conexão com o banco de dados, após, é retornado para o Controlador que envia para a Visão, que exibe as informações ao usuário. Para a comunicação entre o backend e o frontend, foi utilizado uma API REST. O código-fonte do backend está disponível no repositório em: \textit{Backend Encontre Na Ufms} \cite{backend}

\subsubsection{Banco de dados}
    
    Foi utilizado o MySQL, um Sistema de Gerenciamento de Banco de Dados Relacional baseado em SQL, para armazenar e gerenciar os dados do sistema.  Na imagem abaixo, está descrita a estrutura do banco de dados do aplicativo Encontre na UFMS.
    %tambem prefiro o segundo, apagando o texto lá embaixo
    Para o armazenamento de dados no backend, foi utilizado o \textit{MySQL} \cite{mysql}, um sistema de gerenciamento de banco de dados relacional open-source que oferece confiabilidade e desempenho. A conexão entre o Node.js e o MySQL foi feita utilizando o \textit{Drizzle} \cite{drizzle}, um mapeador de objetos (ORM) que foca na velocidade e eficiência na entrega de dados entre as duas ferramentas.

\begin{figure}[h]
    \centering
    \includegraphics[width=0.7\textwidth]{imagens/encontrenaufms.png}
    \caption{\scriptsize Interface do DBeaver exibindo a estrutura do banco de dados \cite{dbeaver}.}
    \label{fig:descricaoBancoDeDados}
\end{figure}

    Como descrito na Figura 6, o centro da aplicação e do banco de dados são locais. A estrutura do banco foi modelada para permitir uma organização eficiente das informações, com tabelas que representam tanto os locais quanto os atributos e relacionamentos associados:

    \begin{itemize}
        \item \textbf{Locales}: Armazena os dados gerais de cada local, como nome, descrição, e localização.
        \item \textbf{Schedules}: Relacionamento de um para um com a tabela Locales, responsável por armazenar os horários de funcionamento dos locais ao longo da semana.
        \item \textbf{Photos}: Relacionamento de um para muitos com a tabela Locales, onde cada local pode ter diversas fotos associadas.
        \item \textbf{Users}: Armazena os dados do usuário, como nome, email, senha.
        \item \textbf{Histories}: Relacionamento de muitos para muitos com a tabela Locales e a tabela Users, onde cada usuário pode ter um histórico de locais visualizados recentemente.
        \item \textbf{Reviews}: Relacionamento de muitos para muitos com a tabela Locales e a tabela Users, armazena as avaliações de cada usuário feitas aos locais.
        \item \textbf{Favorites}: Relacionamento de muitos para muitos com a tabela Locales e a tabela Users, onde cada usuário pode ter locais favoritos.
        \item \textbf{AcademicBlocks}: Tabela exclusiva para armazenar os nomes dos cursos pertencentes ao respectivo bloco acadêmico.
        \item \textbf{Libraries}: Tabela exclusiva para armazenar as as regras e o link para site externo da biblioteca.
        \item \textbf{Sports}: Tabela exclusiva para armazenar as regras e os esportes disponíveis para serem praticados no centro de esportes.
        \item \textbf{Transportes}: Tabela exlusiva para armazenenar os ônibus que passam no local.
    \end{itemize}

    A tabela ``Locales'' armazena os dados gerais que compõem um determinado local. Sendo que cada local pode possuir horários de funcionamento durante a semana, sendo armazenado na tabela ``Schedules''. Os locais também podem ter fotos, sendo armazenados na tabela ``Photos''.
    
    Os locais foram dividos em tipos específicos para melhor organização dos dados e apresentação ao usuário final, sendo eles: Blocos Acadêmicos, Pontos Turísticos, Bancos, Restaurantes, Serviços de Saúde, Bibliotecas, Centros de Esportes, Transportes, Estacionamentos e Prédios Gerais. Alguns desses tipos posssuem tabelas próprias com informações específicas, como é o caso dos tipos: Blocos Acadêmicos, Bibliotecas, Centros de Esportes e Transportes. 
    
    Além disso, também foi criado uma tabela ``Users'' para armazenar os dados básicos do usuário para cadastro e login do mesmo dentro do sistema. Os usuários possuem 3 relações com a tabela ``Locales'', formando 3 tabelas, sendo elas: ``Histories'', ``Reviews'' e ``Favorites'', que armazenam respectivamente:
    
    \begin{itemize}
        \item O histórico dos 5 últimos locais visualizados recentemente
        \item As avaliações que o usuário fez aos locais 
        \item Os locais favoritos.
    \end{itemize}

    Para a conexão entre o Node.js e o MySQL, foi utilizado o Drizzle que é uma ferramenta de mapeamento de objetos, cujo o foco é a velocidade de entrega de dados entre as duas ferramentas.

\subsection{Login}
    Os usuários tem a possibilidade de se cadastrarem no aplicativo, sendo o cadastro um requisito não obrigatório. Para usuários logados no sistema, são utilizados tokens \textit{JWT (JSON Web Token)} \cite{jwt} para autenticação, garantindo acesso seguro e eficaz ao sistema.
    
    Para o envio de e-mails de recuperação de senha, foi utilizado o \textit{MailJet} \cite{mailjet}, uma ferramenta que fornece acesso a APIs para envio de e-mails e mensagens de texto (SMS).

\subsection{Ambiente de desenvolvimento: Visual Studio Code}
    O ambiente de desenvolvimento escolhido foi o \textit{Visual Studio Code} \cite{visualstudiocode}, uma das IDEs mas utilizadas no mundo graças ao grande número de ferramentas que podem ser integradas nela para aumentar as funcionalidades de desenvolvimento, teste e depuração.
    
    Foi utilizado também o \textit{Android SDK (Software Development Kit)} \cite{androidsdk}, que é um conjunto de ferramentas que permite o desenvolvimento de aplicativos para a plataforma Android. Ele inclui um depurador, bibliotecas, um emulador de dispositivo baseado em \textit{QEMU} \cite{QEMU}, documentação, amostras de código e tutoriais.
    
\subsection{Bibiliotecas e Dependências}
    Diversas bibliotecas e dependências foram utilizadas no desenvolvimento do aplicativo Encontre na UFMS, dentre elas destacam-se:
    
    \begin{itemize}
      \item \textbf{Fastify}: Framework para desenvolvimento de aplicações web com Node.js \cite{fastify}.
      \item \textbf{TypeScript}: Linguagem de programação que extende JavaScript, adicionando tipagem à linguagem \cite{typescript}.
      \item \textbf{tsup}: Empacotador de arquivos TypeScript e JavaScript, utilizado para otimizar e gerar builds \cite{tsup}.
      \item \textbf{MySQL}: MySQL é um sistema de gerenciamento de banco de dados relacional de código aberto baseado em SQL \cite{mysql}.
      \item \textbf{Drizzle}: Mapeador de objetos para conexão entre Node.js e MySQL \cite{drizzle}.
      \item \textbf{MailJet}: Ferramenta para envio de emails e sms \cite{mailjet}.
      \item \textbf{Flutter}: Framework para desenvolvimento de aplicativos móveis \cite{flutter}.
      \item \textbf{Dart}: Linguagem de programação utilizada no Flutter \cite{dart}.
      \item \textbf{hive}: Banco de dados não relacional utilizado para armazenamento local \cite{hive}.
      \item \textbf{google\_maps\_flutter}: Biblioteca para implementação de mapas no Flutter \cite{googlemapsflutter}.
      \item \textbf{flutter\_bloc}: Biblioteca para gerenciamento de estados no Flutter \cite{flutterbloc}.
      \item \textbf{dio}: Biblioteca para requisições HTTP no Flutter \cite{dio}.
    \end{itemize}
    
    Todas essas bibliotecas e dependências, bem como outras que foram especificadas nos projetos do frontend e do backend, foram fundamentais para o desenvolvimento do aplicativo Encontre na UFMS, permitindo a implementação de funcionalidades essenciais e a integração entre o backend e o frontend.

\subsection{Testes e Execução}

    O aplicativo Encontre na UFMS foi testado em diversos dispositivos Android, a fim de garantir a compatibilidade e o bom funcionamento em diferentes tamanhos de tela e versões do sistema operacional. Foi utilizado o emulador do Android Studio para testar o aplicativo no Pixel 6a na API 33 do Android mas foi principalmente testado em dispositivos físicos, como o Galaxy S21 fe e Xiaomi Note 8 Pro, pela facilidade de uso e para poder verificar a usabilidade real do aplicativo em todo momento.

\subsubsection{Diagrama de Navegação}

    A Figura 7 ilustra o diagrama de navegação do aplicativo Encontre na UFMS:

    \begin{figure}[h]
        \centering
        \includegraphics[width=1\textwidth]{imagens/navegacao.png}
        \caption{\scriptsize Diagrama de Navegação}
        \label{fig:diagrama-navegacao}
    \end{figure}

    O diagrama de navegação do aplicativo apresentado acima demonstra os fluxos básicos de navegação que o usuário pode experenciar
    durante o uso do mesmo, vale destacar que algumas telas como Perfil e Criação só são acessíveis caso o usuário esteja logado.

    \FloatBarrier

\subsubsection{Tela 1: Listagem de Locais}

    A tela de listagem de locais é a tela inicial do aplicativo, a primeira e principal tela apresentada para o usuário. Nela, o usuário pode visualizar todos os locais cadastrados no aplicativo, podendo filtrar os locais por categoria e buscar locais através de um campo de pesquisa por texto. 
    
    Cada local é representado por um segmento contendo a foto, o nome, o endereço, a categoria, se o local está provavelmente fechado naquele horário e um indicador que mostra se o local está marcado como favorito ou não. Cada segmento é clicável, redirecionando o usuário para a tela de detalhes do local. O usuário pode também clicar no indicador de favorito para marcar ou desmarcar o local como favorito caso ele esteja logado, caso não esteja o usuário é redirecionando para a tela de Login que será abordada mais a frente. A listagem de locais é paginada, exibindo 10 locais por vez, uma nova requisição é feita toda vez que o usuário chega ao final da lista, carregando a próxima página, se houver.

    Na parte superior da tela há uma seção com um botão, que abre o menu lateral, e um campo de texto para pesquisa. Logo abaixo há um filtro de categorias, uma série de botões que podem ser arrastados horizontalmente, permitindo que o usuário filtre os locais por uma ou mais categorias, de acordo com suas necessidades. Na busca por texto foi aplicado um \textit{debounce} de 1s para evitar que a busca seja feita a cada caractere digitado, o que poderia causar um consumo excessivo de recursos da API. A Figura 8 mostra a tela de listagem de locais.

    \begin{figure}[h]
        \centering
        \includegraphics[width=0.4\textwidth]{imagens/inicial.jpg}
        \caption{\scriptsize Tela 1: Listagem de Locais}
        \label{fig:tela1}
    \end{figure}

    \FloatBarrier

\subsubsection{Tela 2: Local}

    A tela de local apresenta todos os detalhes do local que foi selecionado pelo usuário. Nela, o usuário pode visualizar uma ou mais fotos do local, o nome, um indicador que é mostrado caso o local tenha opções de acessibilidade e a avaliação média do local, que é a média das avaliações por outros usuários. Além disso existem 3 abas que o usuário pode navegar sem sair da tela. A aba de localização é a aba inicial e mostra o endereço do local, um mapa com a localização do local indicada e uma seção para o usuário dar sua própria avaliação do local, caso esteja logado. A aba de informações exibe alguns detalhes do local como observações e telefone de contato. Por fim, a aba de Horários exibe os horários de funcionamento do local, caso estejam cadastrados. A Figura 9 mostra a tela de local com a aba de localização, já a Figura 10 apresenta as demais abas.

    \begin{figure}[h]
        \centering
        \includegraphics[width=0.4\textwidth]{imagens/local.jpg}
        \caption{\scriptsize Tela 2: Local}
        \label{fig:tela2}
    \end{figure}

    \begin{figure}
        \centering
        \includegraphics[width=0.4\textwidth]{imagens/info.jpg}
        \hspace{10mm}
        \includegraphics[width=0.4\textwidth]{imagens/horario.jpg}
        \caption{\scriptsize Abas: Informações e Horários}
        \label{fig:tela2-abas}
    \end{figure}

    \FloatBarrier

\subsubsection{Tela 3: Menu Lateral}

    O menu lateral é acessível através de um botão na parte superior da tela de listagem de locais. Ele é composto por uma série de botões que redirecionam o usuário para diferentes telas do aplicativo. O menu lateral é uma forma de organizar as funcionalidades do aplicativo de forma intuitiva e acessível, permitindo que o usuário navegue facilmente entre as diferentes telas do aplicativo.

    Essa tela muda de acordo com o estado de autenticação do usuário, caso ele esteja logado, aparecerá o nome do usuário logo abaixo o nome do aplicativo e os botões de Perfil e Criação assim como um botão de Sair, caso contrário, todas essas informações são omitidas e apenas o botão de Sobre é exibido, e no lugar do botão de Sair é exibido o botão de Login. Abaixo as duas figuras exemplificam ambos os estados. A Figura 11 mostra o menu lateral tanto no estado logado quanto no estado deslogado.

    \begin{figure}[h]
        \centering
        \includegraphics[width=0.4\textwidth]{imagens/menu-lateral-logado.jpg}
        \hspace{10mm}
        \includegraphics[width=0.4\textwidth]{imagens/menu-lateral-logout.jpg}
        \caption{\scriptsize Tela 2: Menu Lateral (Logado e Deslogado)}
        \label{fig:tela2-logado}
    \end{figure}

    \FloatBarrier

\subsubsection{Tela 4: Perfil}

    A tela de perfil exibe as informações do usuário logado, nome e e-mail. Além disso, a tela também permite que o usuário edite seu nome e altere sua senha. A tela de perfil é uma forma de o usuário gerenciar suas informações pessoais e garantir que seus dados estejam sempre atualizados. A Figura 12 mostra a tela de perfil.

    \begin{figure}[h]
        \centering
        \includegraphics[width=0.4\textwidth]{imagens/perfil.jpg}
        \caption{\scriptsize Tela 4: Perfil}
        \label{fig:tela4}
    \end{figure}

    \FloatBarrier

\subsubsection{Tela 5: Criação}

    A tela de criação possibilita que os usuários preencham uma série de informações sobre o local que desejam cadastrar, como nome, endereço, categoria, horários de funcionamento, telefone de contato, observações e fotos, e enviem para uma avaliação. A tela de criação é uma forma de os usuários contribuírem com o aplicativo, adicionando novos locais e enriquecendo a base de dados do sistema. A Figura 13 mostra as telas de criação.

    \begin{figure}[h]
        \centering
        \includegraphics[width=0.4\textwidth]{imagens/criacao.jpg}
        \hspace{10mm}
        \includegraphics[width=0.4\textwidth]{imagens/criacao2.jpg}
        \caption{\scriptsize Tela 5: Criação}
        \label{fig:tela5}
    \end{figure}

    \FloatBarrier

\subsubsection{Tela 6: Sobre}

    A tela de sobre exibe informações sobre o aplicativo, como a versão, os desenvolvedores e o código fonte. A tela de sobre é uma forma de os usuários conhecerem mais sobre o aplicativo e os desenvolvedores por trás dele. A Figura 14 mostra a tela de sobre.

    \begin{figure}[h]
        \centering
        \includegraphics[width=0.4\textwidth]{imagens/sobre.jpg}
        \caption{\scriptsize Tela 6: Sobre}
        \label{fig:tela6}
    \end{figure}

    \FloatBarrier

\subsubsection{Tela 7: Login}

    A tela de login pode ser alcançada através de 2 fluxos, o primeiro é através do botão de Login na tela de Menu Lateral e o segundo é quando o usuário tentar marcar um local como favorito sem estar logado. A tela de login é composta por um campo de e-mail e um campo de senha, além de botões para iniciar o processo de registro de uma nova conta ou para recuperar a senha, caso o usuário tenha esquecido. A Figura 15 mostra a tela de login e de cadastro.

    \begin{figure}[h]
        \centering
        \includegraphics[width=0.4\textwidth]{imagens/login.jpg}
        \hspace{10mm}
        \includegraphics[width=0.4\textwidth]{imagens/registrar.jpg} %tela de registro
        \caption{\scriptsize Tela 7: Login}
        \label{fig:tela7}
    \end{figure}

    \FloatBarrier
    
    Caso o usuário entre no processo de recuperação de senha ele deverá informar o email cadastrado e um email será enviado com um código de recuperação, que deverá ser informado numa tela subsequente, caso a verificação seja bem sucedida o usuário poderá definir uma nova senha. A Figura 16 mostra as telas de recuperação de senha.
    
    \FloatBarrier

    \begin{figure}[h]
        \centering
        \includegraphics[width=0.4\textwidth]{imagens/email.jpg} %tela de email
        \hspace{10mm}
        \includegraphics[width=0.4\textwidth]{imagens/nova-senha.jpg} %tela de codigo
        \caption{\scriptsize Tela 7: Login}
        \label{fig:tela7-recuperacao}
    \end{figure}

    \FloatBarrier