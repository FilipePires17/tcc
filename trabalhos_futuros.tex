\section{Trabalhos Futuros}
\label{sec:trabalhos_futuros}

    Por mais que o aplicativo já seja funcional, vemos que ainda há muito espaço para melhorias e novas funcionalidades. A seguir, listamos algumas ideias para trabalhos futuros:
    
    Requisitos que foram levantados mas não implementados seriam um bom ponto de partida para futuras implementações. Alguns exemplos são a possibilidade de se registrar e entrar no aplicativo através de serviços externos como o Google e o Facebook, o histórico de pesquisa para facilitar a experiência do usuário, a introdução de um modo claro, a adição de opções de acessibilidade mais avançadas e também um recurso para que o usuário possa reportar problemas com o aplicativo.
    
    Seria interessante adicionar uma seção de comentários nos locais, onde os usuários pudessem deixar suas opiniões e avaliações mais detalhadas sobre o funcionamento do local. Isso poderia para serviços como o de restaurantes e lanchonetes, por exemplo.

    Por mais que o aplicativo possa ser considerado como pronto para uso, ele ainda não foi lançado oficialmente. Seria importante que fosse disponibilizado para download em lojas de aplicativos, como a Google Play Store, isso também exigiria que o backend fosse hospedado em um servidor para que o aplicativo pudesse ser acessado a qualquer momento.

    Atualmente as sugestões de novos locais, assim como as de alterações, são enviadas por email para análise do administrador. Criar um serviço web simples que permita a visualização e aprovação dessas sugestões com maior facilidade e clareza ajudaria muito o processo, o que aceleraria a adição de novos locais e a correção de informações erradas.

    Essas melhorias fariam o Encontre na UFMS mais completo e agradável de se usar, o que consolidaria seu uso e popularidade entre os estudantes e visitantes da UFMS.