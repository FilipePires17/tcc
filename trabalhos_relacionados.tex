\section{Trabalhos Relacionados}

Nesta seção abordamos aplicativos existentes que têm objetivos semelhantes ao do aplicativo Encontre na UFMS e que podem fornecer insights úteis para seu desenvolvimento e implementação. Dois aplicativos relevantes nesse contexto são \textit{Google Maps}\cite{maps2005} e \textit{Localização UFMS}\cite{localizacaoufms}.

O Google Maps, da Google LLC, é altamente utilizado no mundo todo diariamente como ferramenta de geolocalização. Ele possui diversas funcionalidades, sendo uma das principais a de encontrar um local desejado pelo usuário e, também, fornecer uma rota viável para que o usuário possa chegar ao destino. O aplicativo da Google também é capaz de dar diversas informações sobre os locais disponíveis para busca, como horário de funcionamento e informações de contato.

O Localização UFMS, desenvolvido pela própria UFMS, é uma aplicação web que disponibiliza um mapa interativo de todos os campi da UFMS, incluindo diversos pontos de interesse. Nessa ferramenta, o usuário consegue realizar buscas utilizando diversos filtros e visualizar informações detalhadas sobre os pontos de interesse, como lotação, horários, disponibilidade e agenda. Dependendo da classificação de um certo local pode mostrar informações diferentes, alguns possuem até mesmo fotos enquanto outros só possuem informações de tamanho do local. 