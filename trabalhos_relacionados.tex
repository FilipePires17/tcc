\section{Trabalhos Relacionados}
\label{sec:trabalhos_relacionados}

Nesta seção abordamos aplicativos existentes que têm objetivos semelhantes ao do aplicativo Encontre na UFMS e que podem fornecer insights úteis para seu desenvolvimento e implementação. Dois aplicativos relevantes nesse contexto são \textit{Google Maps} \cite{maps2005} e \textit{Localização UFMS} \cite{localizacaoufms}.

O Google Maps, da Google LLC, é altamente utilizado no mundo todo diariamente como ferramenta de geolocalização. Ele possui diversas funcionalidades, sendo uma das principais a de encontrar um local desejado pelo usuário e, também, fornecer uma rota viável para que o usuário possa chegar ao destino. O aplicativo da Google também é capaz de dar diversas informações sobre os locais disponíveis para busca, como horário de funcionamento e informações de contato.

O Localização UFMS, desenvolvido pela própria UFMS, é uma aplicação web que disponibiliza um mapa interativo de todos os campi da UFMS, incluindo diversos pontos de interesse. Nessa ferramenta, o usuário consegue realizar buscas utilizando diversos filtros e visualizar informações detalhadas sobre os pontos de interesse, como lotação, horários, disponibilidade e agenda. Dependendo da classificação de um certo local pode mostrar informações diferentes, alguns possuem até mesmo fotos enquanto outros só possuem informações de tamanho do local.

\subsection{Análise Crítica e contribuição do Encontre na UFMS}
    Para podermos analisar o diferencial do aplicativo Encontre na UFMS podemos compará-lo com os aplicativos Google Maps e Localização UFMS. O Google Maps é um aplicativo de geolocalização muito utilizado no mundo todo, porém, ele não é focado em um local específico, como o campus da UFMS, e sim em todo o mundo, sendo assim ele não possui registrado diversos locais de menor escala mas que podem ser de grande importância para algum visitante ou alunos. Já o Localização UFMS é um aplicativo web que disponibiliza um mapa interativo da UFMS que possui mais de 2900 pontos de interesse registrados, porém, ele não possui alguns pontos de interesse que podem ser de grande importância para a comunidade acadêmica, como por exemplo, o Restaurante Universitário e também não informa ao usuário como chegar até o local desejado já que ele é focado em apenas mostrar informações sobre um certo local, ele também não tem fotos de todos os locais e não é colaborativo uma vez que apenas a administração da UFMS pode adicionar novos locais.

\subsection{Comparação entre o Encontre na UFMS e outros aplicativos}
    A Tabela 1 apresenta uma comparação entre os aplicativos Google Maps, Localização UFMS e Encontre na UFMS, destacando as principais características de cada aplicativo.

\FloatBarrier

\begin{table}[h]
    \begin{tabularx}{\textwidth}{|X|X|X|X|}
        \hline
        \textbf{Características} & \textbf{Google Maps} & \textbf{Localização UFMS} & \textbf{Encontre na UFMS} \\ \hline
        \textbf{Objetivo Principal} & Geolocalização global & Mapa interativo da UFMS & Navegação pelo campus da UFMS \\ \hline
        \textbf{Público-Alvo} & Usuários em geral & Comunidade acadêmica da UFMS & Comunidade acadêmica da UFMS \\ \hline
        \textbf{Principais Funcionalidades} & Busca de locais, rotas, informações de contato, horários de funcionamento & Busca de locais no campus, informações sobre pontos de interesse & Busca de locais, rotas, cadastro colaborativo de locais, avaliações de locais \\ \hline
    \end{tabularx}
    \caption{Comparação entre aplicativos de localização}
    \label{tab:comparacao-aplicativos}
\end{table}