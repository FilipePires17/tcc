\section{Trabalhos Relacionados}

Na seção de trabalhos relacionados, abordamos aplicativos existentes que têm objetivos semelhantes ao do aplicativo Encontre na UFMS e que podem fornecer insights úteis para seu desenvolvimento e implementação. Dois aplicativos relevantes nesse contexto são \textit{Google Maps}\cite{maps2005} e \textit{Localização UFMS}\cite{localizacaoufms}.

O Google Maps, da Google LLC, é altamente utilizado no mundo todo diariamente como ferramenta de geolocalização. Ele possui diversas funcionalidades, uma das principais é encontrar um local desejado pelo usuário e também informar uma rota viável para que o usuário possa chegar em tal localidade. O aplicativo da Google também é capaz de dar diversas informações sobre os locais disponíveis para busca, como horário de funcionamento e informações de contato.

O Localização UFMS, desenvolvido pela própria UFMS, é uma aplicação web que disponibiliza um mapa interativo da UFMS, todos os campi, com muitos pontos de interesse. Nessa ferramenta o usuário consegue realizar buscas com diversos filtros e ver diversas informações sobre os pontos de interesse como lotação, horários, disponibilidade e agenda. Dependendo da classificação de um certo local pode mostrar informações diferentes, alguns possuem até mesmo fotos enquanto outros só possuem informações de tamanho do local. 